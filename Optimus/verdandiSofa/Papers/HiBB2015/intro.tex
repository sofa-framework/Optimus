\section{Introduction}
\label{s:intro}
In the last decade the importance of computer medical simulation in surgical training, pre-operative planning and intra-operative guidance has increased considerably. The recent advances show that the physics-based simulations are becoming capable of providing efficient improvements in laparoscopic as well as open surgery: e.g. the augmented-reality techniques employing a patient specific model built from the pre-operative data begin to play an important role in surgical navigation during laparoscopic surgery~\cite{haouchine2015impact}. Further, the medical simulations almost directly contribute to the actual boost in the area of interventional radiology, which provides an alternative for treatment of complicated pathologies. 

However, the successful employment of computer simulation in surgical navigation and planning necessitates patient-specific modeling: 
besides the geometrical aspects which often show significant inter-subject variability, correct patient-specific parametrization of the physical models is another crucial condition necessary to maintain required accuracy, reliability and robustness of models. Nevertheless, despite recent advances in non-invasive techniques such as MRI and ultrasound elastography, parametrization of tissue models remain a difficult and challenging task. 

One way of addressing the issue is to extend the deterministic simulation by methods based on stochastic modeling. The computer model itself thus becomes a part of an iterative process usually referenced as \emph{data assimilation} where the numerical model is employed in the prediction phase, while the observations of the simulated phenomena are used to correct the estimations of the parameters as well as the predicted~\cite{grewal2014kalman}. Naturally, when compared to a  the \emph{forward} simulation executed with given parametrization, the data assimilation methods based on the predictor-corrector schemes repeatedly invoking 
the model function for different parametrizations in each step are computationally much more expensive.

While the data assimilation techniques have been usually employed to deal with systems having a low number of both parameters and degrees of freedom (DoF), recently published methods such as reduced-order Kalman filtering~\cite{moireau2011reduced} open new possibilities allowing for the data assimilation to be applied to models with significantly larger number of parameters and DoFs. The novel methods open new horizons in stochastic modeling: a typical example of a possible scenario is an image-guided navigation of a surgical intervention such as a laparoscopic hepatectomy or partial nephrectomy, where the actual position of the tumor is predicted via physics-based simulation using the pre-operative data and intra-operative acquisitions represented for example by a flow captured by a laparoscopic camera or 2D slice scanned by an ultrasound probe. Since the accuracy of the prediction of the tumor position depends on a model which in turn depends on the tissue parameters such as elasticity of the tissue which moreover display high level of heterogeneity, data assimilation seems to be a promising tool combining the computer model with corresponding uncertainties related to its parametrization and the real procedure integrated into the correction process via observations.

Indeed, the scenario described above requires tremendously efficient methods of both the computer simulation and data assimilation, ideally achieving real-time 
performance. Although the existing methods are still far from this ambitious objective, novel mathematical methods in combination with 
advanced technologies of computer science such as parallelization and hardware acceleration already bring encouraging results.  

In this rather technical paper, we focus on the implementation of a data assimilation technique based on the reduced-order Kalman filtering for estimation of parameters of a soft-tissue model based on the finite element formulation of non-linear elasticity. The implementation is based on an integration of two state-of-the-art packages: the Simulation Open Framework Architecture (SOFA) which is an open-source package implementing advanced physics-based modeling methods focusing on real-time simulation in medicine, and Verdandi, an open-source library implementing the recently introduced data-assimilation schemes. We also describe the parallelization which is employed in order to speed up the prediction phase.

After giving the implementation details, we present a set of preliminary results demonstrating the capabilities of the new data assimilation framework. Using 
synthetic data for which the ground truth is known, we first assess the accuracy of the parameter estimation in two different scenarios. Then, we evaluate the performance of the actual implementation and report the impact of further optimizations based on parallelization. Finally, we discuss the results and provide further perspectives which should bring the data-assimilation of models with large number of parameters closer to the real-time scenario required in many applications related to image-guided and simulation-based intra-operative navigation.
